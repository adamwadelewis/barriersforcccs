\documentclass{article}

\usepackage[
  paperheight=8.5in,
  paperwidth=5.5in,
  left=10mm,
  right=10mm,
  top=20mm,
  bottom=20mm]{geometry}
\usepackage[utf8]{inputenc}

\usepackage{graphicx}
\usepackage{wrapfig}
\usepackage[bottom]{footmisc}
\usepackage{listings}
\usepackage{enumitem}

\usepackage{wrapfig}
\usepackage{ragged2e}

\usepackage{array}
\usepackage[table]{xcolor}
\usepackage{multirow}
\usepackage{booktabs}
\usepackage{hhline}
\definecolor{palegreen}{rgb}{0.6,0.98,0.6}

\usepackage{amsmath}
\usepackage{amssymb}
\usepackage{multicol}
\usepackage{lipsum}
\usepackage{hyphenat}
\PassOptionsToPackage{hyphens}{url}
\usepackage{url}

\usepackage{rotating}

%\usepackage{xeCJK}

%% support use of straight quotes in code listings
\usepackage[T1]{fontenc}
\usepackage{textcomp}
\usepackage{listings}
\lstset{upquote=true}

%% for shrinking space between lines
\usepackage{setspace}

\newcommand*{\affaddr}[1]{#1} % No op here. Customize it for different styles.
\newcommand*{\affmark}[1][*]{\textsuperscript{#1}}
\newcommand*{\email}[1]{\small{\texttt{#1}}}
\newcommand{\tarot}{\textsc{Tarot}}
\renewcommand*\contentsname{\centering Table of Contents}

\renewcommand{\footnoterule}{%
  \kern -3pt
  \hrule width \textwidth height 0.5pt
  \kern 2pt
}

% remove date
\date{}

\usepackage{titlesec}
\titleformat*{\section}{\large\bfseries}
\titleformat*{\subsection}{\normalsize\bfseries}
\titleformat*{\subsubsection}{\normalsize\bfseries}


\title{Addressing Barriers for Transfer Into University Computer Science
  Programs for Community College
  Students\footnote{\protectCopyright is held by the author/owner.
}
\\
\vspace{0.2in}
\large Panel Discussion
}

\author{
Cara Tang\affmark[1], Adam Wade Lewis\affmark[2], Karen Works\affmark[3]\\
\affmark[1]Computer Information Systems\\
Portland Community College, Portland, OR 97229\\
\email{cara.tang@pcc.edu}\\
\affmark[2]Department of Mathematical, Computer, and Natural Sciences,\\
College of Arts and Sciences,\\
Athens State University, Athens, AL 35611\\
\email{Adam.Lewis@athens.edu}\\
Computer Science Department\\
College of Arts and Sciences\\
Florida State University-Panama City, FL 32405\\
\email{keworks@fsu.edu}
}

\begin{document}
\maketitle

\section{Summary}
Transferring from a community college to a university computer science
program presents many barriers to the student.  For many  students
the barriers are raised higher by the necessity of having to take a
hiatus from school to address family and work responsibilities. Students
face challenges in the transition back into the academic environment
after many years.

Academically, students transferring from the community college face
barriers related to credit transfer, course equivalency, and
prerequisite gaps. Without clear articulation agreements, students face
the problem of not all credits transferring, especially specialized or
technical courses that have no matching university courses. Students
often face concerns regarding course equivalency with perceived
differences in course content and rigor.

Adult learners have been identified as an important under represented  market segment
in higher education as small institutions struggle to address questions
of financial exigency and demographic changes in the population.
Institutions need to understand the demographics and needs of this group
of students is very different than the 18-22 year-old age group of
traditional college students.

\section{Cara Tang}

\section{Adam Wade Lewis}
Athens State University is a two hundred year old institution that has
been an upper-division only university for over fifty years.
So the needs of the adult learner and transfer student are paramount to
our institution.   The care and support of these students is very
different than the usual population of 18-22 year-old students that form
the traditional body of college students recruited by our institutions.

For the computing disciplines, our efforts at Athens State University
have been deeply focused on the creation and tuning of articulation
agreements.  This process is complicated by the focus in our community
college partners on terminal associate degrees and their pivot towards
certificates and micro-credentials.  Students transferring with those
credentials find themselves with many credit hours that do not transfer
and often discover the unfortunate need for two to four additional
semesters to address what we call the ``general education impedance
mismatch''.

This is why the work on curriculum guidelines happening
within organizations such as ACM2Y is so
important.  These organizations are putting great effort into
harmonizing the curriculum guidelines between the two year and four year
institutions. Understanding and aiding this effort is critical to any
pivot towards better serving adult learners.

\section{Karen Works}

\section{Biographies}
\textbf{Cara Tang} is a faculty member in the Computer Information
Systems Department at Portland Community College in Porland, Oregon.   They are the
national chair the ACM2Y special interest group on computing education
in two-year programs and is the Past Chair of the ACM's Committee for
Computing Education in Community Colleges and a Member of the ACM
Education Board.

\noindent
\textbf{Adam Wade Lewis} is an Associate Professor of Computer Science
and Program Coordinator for Computer Science and Information Technology
at Athens State University, in Athens, Alabama.  They are actively
involved with transfer advising and academic advising process working
with both community college transfer students and transfer students from
other senior institutions.

\noindent
\textbf{Karen Works} is an Assistant Teaching Professor of Computer
Science in the Computer Science Department at the Panama City campus of
Florida State University.  

\end{document}
