\documentclass{article}

\usepackage[
  paperheight=8.5in,
  paperwidth=5.5in,
  left=10mm,
  right=10mm,
  top=20mm,
  bottom=20mm]{geometry}
\usepackage[utf8]{inputenc}

\usepackage{graphicx}
\usepackage{wrapfig}
\usepackage[bottom]{footmisc}
\usepackage{listings}
\usepackage{enumitem}

\usepackage{wrapfig}
\usepackage{ragged2e}

\usepackage{array}
\usepackage[table]{xcolor}
\usepackage{multirow}
\usepackage{booktabs}
\usepackage{hhline}
\definecolor{palegreen}{rgb}{0.6,0.98,0.6}

\usepackage{amsmath}
\usepackage{amssymb}
\usepackage{multicol}
\usepackage{lipsum}
\usepackage{hyphenat}
\PassOptionsToPackage{hyphens}{url}
\usepackage{url}

\usepackage{rotating}

%\usepackage{xeCJK}

%% support use of straight quotes in code listings
\usepackage[T1]{fontenc}
\usepackage{textcomp}
\usepackage{listings}
\lstset{upquote=true}

%% for shrinking space between lines
\usepackage{setspace}

\newcommand*{\affaddr}[1]{#1} % No op here. Customize it for different styles.
\newcommand*{\affmark}[1][*]{\textsuperscript{#1}}
\newcommand*{\email}[1]{\small{\texttt{#1}}}
\newcommand{\tarot}{\textsc{Tarot}}
\renewcommand*\contentsname{\centering Table of Contents}

\renewcommand{\footnoterule}{%
  \kern -3pt
  \hrule width \textwidth height 0.5pt
  \kern 2pt
}

% remove date
\date{}

\usepackage{titlesec}
\titleformat*{\section}{\large\bfseries}
\titleformat*{\subsection}{\normalsize\bfseries}
\titleformat*{\subsubsection}{\normalsize\bfseries}


\title{Addressing Barriers for Transfer Into University Computer Science
  Programs for Community College
  Students\footnote{\protectCopyright is held by the author/owner.
}
\\
\vspace{0.2in}
\large Panel Discussion
}

\author{
Cara Tang\affmark[1], Adam Wade Lewis\affmark[2], Karen Works\affmark[3]\\
\affmark[1]Computer Information Systems\\
Portland Community College, Portland, OR 97219\\
\email{cara.tang@pcc.edu}\\
\affmark[2]Department of Mathematical, Computer, and Natural Sciences,\\
College of Arts and Sciences,\\
Athens State University, Athens, AL 35611\\
\email{Adam.Lewis@athens.edu}\\
Computer Science Department\\
College of Arts and Sciences\\
Florida State University-Panama City, FL 32405\\
\email{keworks@fsu.edu}
}

\begin{document}
\maketitle

\section{Summary}
Transferring from a community college to a university computer science
program presents many barriers to the student. Many  students face
personal barriers balancing education with family and work
responsibilities.  Some of these students wait before continuing on and
face challenges in the transition back into the academic environment
after many years.

Academically, students transferring from the community college face
barriers related to credit transfer, course equivalency, and
prerequisite gaps. Without clear articulation agreements, students face
the problem of not all credits transferring, especially specialized or
technical courses that have no matching university courses. Students
often face concerns regarding course equivalency with perceived
differences in course content and rigor.

Adult non-traditional students are an important under represented  market segment
in higher education as small institutions struggle to address questions
of financial exigency and demographic changes in this population.
Institutions need to understand the demographics and needs of this group
of students are very different than the 18-22 year-old age group of
traditional college students.

\section{Cara Tang}
Portland Community College (PCC) is the largest institution of higher
education in the state of Oregon and offers a variety of computing
education programs, including both those dedicated to preparing students
for transfer into a Bachelor's program (Computer Science), and those
designed to help students get a job (Computer Information Systems,
Cybersecurity, Network Administration, and Web Development and
Design). However, even many students in the career-oriented programs
intend to transfer to a four-year institution. The ideal transfer
pathway is a 2 + 2, where a student spends two years at the community
college, likely receiving an Associate's degree, and then two years at a
four-year college or university, receiving a Bachelor's degree after a
total of four years. Unfortunately, a 2 + 2 is more the exception than
the rule.

PCC's Computer Science program is designed to align with Portland State
University's Computer Science program, and theoretically it offers a
smooth transfer path. However, since many PCC students require
additional math, reading, or writing classes, the pathway may be a 3 + 2
instead of a 2 + 2. A further stress point is that PCC's program does
not align as well with CS programs at other Oregon institutions students
may be interested in, such as Oregon State University and the University
of Oregon.

On the other hand, PCC's career-oriented computing programs are highly
technical, with the goal of giving students enough skills in two years
to get a job in the IT industry, while at the same time, aligning with
as many local and regional four-programs as possible in order to keep
the door open for students who may choose to pursue a Bachelor's
degree. These career-oriented programs have a higher technical component
and a lower general education component than transfer programs, and
often result in a 2 + 3 model for students who choose to transfer.

Despite many barriers, a positive trend in support of community college
transfer students in career-oriented computing programs has been the
growing number of Bachelor of Applied Science (BAS) programs in
computing disciplines. These programs tend to accept more of the credits
that students obtain at the community college, and in some cases, even
accept an Associate of Applied Science degree as a block of two years
worth of credit, giving a transfer student immediate junior standing.

\section{Adam Wade Lewis}
Athens State University is a two hundred year old institution that has
been an upper-division only university for over fifty years.
So the needs of the adult learner and transfer student are paramount to
our institution.   The care and support of these students is very
different than the usual population of 18-22 year-old students that form
the traditional body of college students recruited by our institutions.

For the computing disciplines, our efforts at Athens State University
have been deeply focused on the creation and tuning of articulation
agreements.  This process is complicated by the focus in our community
college partners on terminal associate degrees and their pivot towards
certificates and micro-credentials.  Students transferring with those
credentials find themselves with many credit hours that do not transfer
and often discover the unfortunate need for two to four additional
semesters to address what we call the ``general education impedance
mismatch''.

This is why the work on curriculum guidelines happening
within organizations such as ACM2Y is so
important.  These organizations are putting great effort into
harmonizing the curriculum guidelines between the two year and four year
institutions. Understanding and aiding this effort is critical to any
pivot towards better serving adult learners.

\section{Karen Works}
The FSU online Computer Science program is an ABET accredited program which accepts students who have completed a minimum of 52 hours of credit at FSU, or an A.A degree. We serve a diverse nontraditional student population and support students around the globe.
 
 Beyond the traditional issues of articulation agreements between schools, we have the additional requirement of ensuring that our transfer students are prepared for the rigor of an ABET accredited program. We work closely with many community colleges and universities throughout the country to ensure that these students make successful transitions.
 
FSU is dedicated to our non-traditional online students and has
developed many resources to ensure that anything a face to face student
can do is available for our online students as well.

\section{Biographies}
\textbf{Cara Tang} is a faculty member in the Computer Information Systems Department at Portland Community College in Portland, Oregon, and leads PCC's Cybersecurity program. She is the Chair of ACM2Y, an ACM group focusing on computing education in two-year programs; Past Chair of the ACM’s Committee for Computing Education in Community Colleges; and a Member of the ACM Education Board.

\noindent
\textbf{Adam Wade Lewis} is an Associate Professor of Computer Science
and Program Coordinator for Computer Science and Information Technology
at Athens State University, in Athens, Alabama.  They are actively
involved with transfer advising and academic advising process working
with both community college transfer students and transfer students from
other senior institutions.

\noindent
\textbf{Karen Works} is an Assistant Teaching Professor of Computer
Science in the Computer Science Department at the Panama City campus of
Florida State University.  

\end{document}

%%% Local Variables:
%%% mode: latex
%%% TeX-master: t
%%% TeX-auto-save: t
%%% TeX-parse-self: t
%%% TeX-PDF-mode: t
%%% End:
